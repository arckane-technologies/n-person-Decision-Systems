%
% DEMOCRATIC SYSTEMS
%
\section{Game Theory and n-person Decision Systems}\label{game theory and n-person decision systems}

\paragraph{} The savvy reader may have noticed that decision systems are fundamentally a game from game theory, a decision system can be mapped to a game in the next manner:

\paragraph{Members} the most trivial transformation is letting every member be a player in the game.

\paragraph{Decisions and ideologies} the real objective of members is to pass decisions that are supported by their own ideology independently of the influence they have, this may be transformed to strategies and paybacks, for any given decision to be made a sub game is created with two possible strategies, voting against or voting in favor, paybacks being a quantity given by how much the ideology supports the logic implications of the decision being passed or being rejected.

\paragraph{Dynamic systems} can be modeled as also subgames, where at a given moment members have the possible strategies of transferring influence to other members under the rules given by the functions of the system. The paybacks are probabilities of future decision passing that favors their ideologies and caused by giving influence to the members they voted for.